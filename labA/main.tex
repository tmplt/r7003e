\documentclass[11pt]{article} % ~~~~~~~~~~~~~~~~~~~~~~~ %
%														%
\input{./Scripts/packages}								%
\input{./Scripts/ridefinitions}							%
\input{./Scripts/figuresgraphicalsettings}				%
\input{./Scripts/tablesgraphicalsettings}				%
\input{./Scripts/newcommands}							%
\input{./Scripts/colors}								%
%														%
% ~~~~~~~~~~~~~~~~~~~~~~~~~~~~~~~~~~~~~~~~~~~~~~~~~~~~~ %

\usepackage{mathtools}

\title{\Huge R7003E 2017 LP2 \\ Lab A report \\ Group 3}
\date{\today}
\author{                        % In alphabetical order, family name first
  Andersson Tommy \\
  Graden Samuel \\
  Sonesten Viktor
}


\begin{document}
\maketitle

\begin{instructions}
	%
	For this lab report there is no size limit on your report, but try to be concise. As for the check-list, put a \checkmark when you have done the indicated things. Before submitting the report you mush have \checkmark-marked every row. Delete also these instructions and all the footnotes before submitting.
	%
\end{instructions}

\section*{Check-list}

\begin{center}
	\rowcolors{1}{black!05!white}{}
	\begin{tabular}{|m{0.8\columnwidth}>{\centering \arraybackslash}m{0.1\columnwidth}|}
		\hline
		the fonts in the figures are the same fonts in the rest of the text\footnote{Suggestion: if you use Matlab then take a look at \texttt{http://staff.www.ltu.se/{\textasciitilde}damvar/matlab.html}. If you instead want to do things seriously in \LaTeX\ then consider using the package \texttt{pgfplots}.} & \\
		the captions of the figures are self-readable and terminate with a dot & \\
		the figures are color-blind-people-friendly and do not have useless backgrounds & \\
		colors are used to convey information\footnote{And not to make figures ``more colored''. E.g., red should be used to indicate something ``wrong'' or ``important'', green something ``correct'', etc.} & \\
		all the figures that are not photos are in vectorial format & \\
		the figures have meaningful legends that allow to understand what is what & \\
		all the acronyms are defined properly\footnote{Suggestion: use the package \texttt{acronym}.} & \\
		there are no spelling errors & \\
		what needs to be referenced is referenced & \\
		the parentheses in the various equations are as big as needed & \\
		the fonts used in the equations are sufficiently big to be readable comfortably & \\
		\hline
	\end{tabular}
\end{center}


\begin{acronym}[TDMA]
	\acro{EOM}	[EOM]	{Equations of Motion}
	\acro{DC}	[DC]	{Direct Current}
	\acro{SS}	[SS]	{State-Space}
	\acro{PID}	[PID]	{Proportional Integrative Derivative}
	\acro{LQR}	[LQR]	{Linear-Quadratic Regulator}
	\acro{emf}	[emf]	{electromotive force}
\end{acronym}
% basic usage: \ac{EOM} \acf{EOM} \acl{EOM}

\subsection*{Reporting of Task 3.1.5}
We derived the following \ac{EOM} for the wheel and body:

\begin{equation}\label{eq:system-init}
  \begin{cases}
    \ddot{x}_w = \frac{F_t - F_x}{m_w}, \\[1em]
    \ddot{\theta}_b =
    \frac{1}{I_b}\left[
      T_f
      - T_m
      + F_y l_b \sin(\theta_b)
      - F_x l_b \cos(\theta_b)
    \right],
  \end{cases}
\end{equation}
where
\begin{equation}\label{eq:F_xyt}
  \begin{cases}
    F_x = m_b\left(
      \ddot{x}_w
      + \ddot{\theta}_b l_b \cos(\theta_b)
      - \dot{\theta}^2_b l_b \sin(\theta_b)
    \right), \\[1em]
    F_y = m_b g+ m_b \ddot{y}_w = m_b g - m_b\left(
      \ddot{\theta}_b l_b \sin(\theta_b)
      + \dot{\theta}^2_b l_b \cos(\theta_b)
    \right), \\[1em]
    F_t = \frac{T_m - T_f - I_w \ddot{\theta}_w}{l_w} = \frac{T_m - T_f}{l_w} - \frac{I_w}{l^2_w} \ddot{x}_w.
  \end{cases}
\end{equation}
We do not want our system equation to depend on $F_x, F_y, F_t$.
Additionally,
\begin{equation}\label{eq:T_f}
T_f = b_f\left(
\dot{\theta}_w - \dot{\theta}_b
\right) =
b_f\left(
\frac{\dot{x}_w}{l_w} - \dot{\theta}_b
\right).
\end{equation}
We substitute Eq.~\eqref{eq:F_xyt}--\eqref{eq:T_f} into Eq.~\eqref{eq:system-init} and simplify:
\begin{equation}\label{eq:system}
  \begin{cases}
    \ddot{x}_w = \frac{1}{m_w + I_w + m_b}\left[
      \frac{T_m - T_f}{l_w}
      + m_b l_b \left(\dot{\theta}^2_b \sin(\theta_b)
        - \ddot{\theta}_b \cos(\theta_b)
      \right)
    \right], \\[1em]
    \ddot{\theta}_b = \frac{1}{I_b + m_b l_b^2}\left[
      b_f\left(\frac{\dot{x}_w}{l_w}
        - \dot{\theta}_b\right)
      - T_m
      + m_b g l_b \sin(\theta_b)
      - \ddot{x}_w m_b l_b \cos(\theta_b)
    \right].
  \end{cases}
\end{equation}

We then find the \ac{EOM} for the motor by applying Newton's laws of rotational motion:
\begin{align}
  J_m \ddot{\theta}_m &= K_t i_m = T_m, \quad \text{assuming negligible viscous coefficient;}\nonumber \\
  \Rightarrow i_m &= \frac{J_m \ddot{\theta}_m}{K_t}\label{eq:motor/newton}
\end{align}
and analyse an equivalent circuit which models the motor:
\begin{equation}\label{eq:motor/circuit}
R_m i_m = v_m - K_e \dot{\theta}_m, \quad \text{assuming negligible inductance.}
\end{equation}
Eq.~\eqref{eq:motor/newton} in Eq.~\eqref{eq:motor/circuit} and rearrangement gives us
\begin{equation}\label{eq:motor/circuit2}
  J_m \ddot{\theta}_m
  + \frac{K_t K_e}{R_m} \dot{\theta}_m
  = \frac{K_t}{R_m}v_m.
\end{equation}
But $J_m \ddot{\theta}_m = T_m$, and $\dot{\theta}_m = \frac{\dot{x}_w}{l_w} - \dot{\theta}_b$ in Eq.~\eqref{eq:motor/circuit2} gives us
\begin{equation}\label{eq:motor}
  T_m = \frac{K_t}{R_m}v_m -
  \frac{K_t K_e}{R_m}\left(
    \frac{\dot{x}_w}{l_w}
    - \dot{\theta}_b
  \right).
\end{equation}
Eq.~\eqref{eq:motor} in Eq.~\eqref{eq:system} gives us our full \ac{EOM} in symbolic form:
\begin{equation}\label{eq:system-nonlinear}
  \begin{cases}
    \begin{aligned}[b]
      \ddot{x}_w =
      \frac{1}{m_w + I_w + m_b}\bigg[
      \frac{K_t}{R_m l_w}\left(
        v_m
        - \frac{K_e \dot{x}_w}{l_w}
        + \dot{\theta}_b
      \right)
      + \frac{b_f \dot{\theta}_b}{l_w}
      - b_f \dot{x}_w\\
      + m_b l_b \left(
        \dot{\theta}^2_b \sin(\theta_b)
        - \ddot{\theta}_b \cos(\theta_b)
      \right)
      \bigg],
    \end{aligned}\\[1em]
  \begin{aligned}[b]
    \ddot{\theta}_b =
    \frac{1}{I_b + m_b l_b^2}\bigg[
      b_f\left(
        \frac{\dot{x}_w}{l_w}
        - \dot{\theta}_b
      \right)
      - \frac{K_t}{R_m l_w}\left(
        v_m
        - \frac{K_e \dot{x}_w}{l_w}
        + \dot{\theta}_b
      \right)\\
      + m_b g l_b \sin(\theta_b)
      - \ddot{x}_w m_b l_b \cos(\theta_b)
    \bigg].
    \end{aligned}
  \end{cases}
\end{equation}

\subsection*{Reporting of Task 3.2.1}
\begin{enumerate}
\item % Say which linearization point you chose;
  We choose the linearization point of $\theta_b = 0$. Thus
  \begin{equation}\label{eq:theta=0}
    \theta_b \approx \dot{\theta}_b \approx \ddot{\theta}_b \approx 0.
  \end{equation}
\item % Say why you chose that specific point;
  We chose $\theta_b = 0$ because it is an equalibrium point and because we assume our system will only be subject to small angles.
\item % Write the linearized EOM
  With small-angle approximation, we redefine our non-linear terms:
  \begin{equation}\label{eq:nonlinear-redef}
    \begin{cases}
      \begin{aligned}
        \sin(\theta_b) &\triangleq \theta_b, \\
        \ddot{x}_w \cos(\theta_b) &\triangleq \ddot{x}_w, \\
        \ddot{\theta}_b \cos(\theta_b) &\triangleq 0, \\
        \dot{\theta}_b^2 \sin(\theta_b) &\triangleq 0.
      \end{aligned}
    \end{cases}
  \end{equation}
  Our linearized \ac{EOM}: Eqs.~\eqref{eq:theta=0}, \eqref{eq:nonlinear-redef}
  in Eq.~\eqref{eq:system-nonlinear} simplifed thus becomes
  \begin{equation}
    \begin{cases}
      \begin{aligned}[b]
        {x}_w \triangleq
        \frac{1}{m_w + I_w + m_b}\left[
          \frac{K_t}{R_m l_w}\left(
            v_m - \frac{K_e \dot{x}_w}{l_w}
          \right)
          - b_f \dot{x}_w
        \right],
      \end{aligned}\\[1em]
      \begin{aligned}[b]
        \ddot{\theta}_b \triangleq
        \frac{1}{I_b + m_b l_b^2}\bigg[
        \frac{b_f \dot{x}_w}{l_w}
        - \frac{K_t}{R_m l_w}\left(
          v_m
          - \frac{K_e \dot{x}_w}{l_w}
        \right)
        - \ddot{x}_w m_b l_b
        \bigg].
      \end{aligned}
    \end{cases}
  \end{equation}
\end{enumerate}

\subsection*{Reporting of Task 3.3.1}
\begin{enumerate}
\item % Say what is your choice for x;
  We define
  \begin{equation}
    u \triangleq v_m, \quad
    y \triangleq \theta_b
  \end{equation}
  \begin{equation}
    \bm{x} \triangleq \begin{bmatrix}
      x_1\\
      x_2\\
      x_3\\
      x_4
    \end{bmatrix}
    \triangleq
    \begin{bmatrix}
      x_w\\
      \dot{x}_w\\
      \theta_b\\
      \dot{\theta}_b
    \end{bmatrix}
    \Rightarrow
    \bm{\dot{x}} =
    \begin{bmatrix}
      x_2\\
      \ddot{x}_w\\
      x_4\\
      \ddot{\theta}_b
    \end{bmatrix}
  \end{equation}
\item % write the matrices A, B, C, and D in (12) in parametric form;
\item % Write the matrices A, B, C, and D in (12) parametric numeric form (for this purpose use Table 4 on page 64).
\end{enumerate}

\end{document}