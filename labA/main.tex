\documentclass[11pt]{article} % ~~~~~~~~~~~~~~~~~~~~~~~ %
%														%
\input{./Scripts/packages}								%
\input{./Scripts/ridefinitions}							%
\input{./Scripts/figuresgraphicalsettings}				%
\input{./Scripts/tablesgraphicalsettings}				%
\input{./Scripts/newcommands}							%
\input{./Scripts/colors}								%
%														%
% ~~~~~~~~~~~~~~~~~~~~~~~~~~~~~~~~~~~~~~~~~~~~~~~~~~~~~ %

\usepackage{mathtools}
\usepackage{subcaption}

\title{\Huge R7003E 2017 LP2 \\ Lab A report \\ Group 3}
\date{\today}
\author{                        % In alphabetical order, family name first
  Andersson Tommy \\
  Graden Samuel \\
  Sonesten Viktor
}


\begin{document}
\maketitle

% \begin{instructions}
% 	%
% 	For this lab report there is no size limit on your report, but try to be concise. As for the check-list, put a \checkmark when you have done the indicated things. Before submitting the report you mush have \checkmark-marked every row. Delete also these instructions and all the footnotes before submitting.
% 	%
% \end{instructions}

\section*{Check-list}

\begin{center}
	\rowcolors{1}{black!05!white}{}
	\begin{tabular}{|m{0.8\columnwidth}>{\centering \arraybackslash}m{0.1\columnwidth}|}
		\hline
		\checkmark the fonts in the figures are the same fonts in the
          rest of the text & \\%\footnote{Suggestion: if you use Matlab then take a look at \texttt{http://staff.www.ltu.se/{\textasciitilde}damvar/matlab.html}. If you instead want to do things seriously in \LaTeX\ then consider using the package \texttt{pgfplots}.} & \\
		\checkmark the captions of the figures are self-readable and terminate with a dot & \\
		\checkmark the figures are color-blind-people-friendly and do not have useless backgrounds & \\
		\checkmark colors are used to convey information & \\% \footnote{And not to make figures ``more colored''. E.g., red should be used to indicate something ``wrong'' or ``important'', green something ``correct'', etc.} & \\
		\checkmark all the figures that are not photos are in vectorial format & \\
		\checkmark the figures have meaningful legends that allow to understand what is what & \\
		\checkmark all the acronyms are defined properly &\\ % \footnote{Suggestion: use the package \texttt{acronym}.} & \\
		\checkmark there are no spelling errors & \\
		\checkmark what needs to be referenced is referenced & \\
		\checkmark the parentheses in the various equations are as big as needed & \\
		\checkmark the fonts used in the equations are sufficiently big to be readable comfortably & \\
		\hline
	\end{tabular}
\end{center}


\begin{acronym}[TDMA]
	\acro{EOM}	[EOM]	{Equations of Motion}
	% \acro{DC}	[DC]	{Direct Current}
	% \acro{SS}	[SS]	{State-Space}
	\acro{PID}	[PID]	{Proportional Integrative Derivative}
	% \acro{LQR}	[LQR]	{Linear-Quadratic Regulator}
	% \acro{emf}	[emf]	{electromotive force}
\end{acronym}
% basic usage: \ac{EOM} \acf{EOM} \acl{EOM}

\subsection*{Reporting of Task 3.1.5}
We derived the following \ac{EOM} for the wheel and body:

\begin{equation}\label{eq:system-init}
  \begin{cases}
    \ddot{x}_w = \frac{F_t - F_x}{m_w}, \\[1em]
    \ddot{\theta}_b =
    \frac{1}{I_b}\left[
      T_f
      - T_m
      + F_y l_b \sin(\theta_b)
      - F_x l_b \cos(\theta_b)
    \right],
  \end{cases}
\end{equation}
where
\begin{equation}\label{eq:F_xyt}
  \begin{cases}
    F_x = m_b\left(
      \ddot{x}_w
      + \ddot{\theta}_b l_b \cos(\theta_b)
      - \dot{\theta}^2_b l_b \sin(\theta_b)
    \right), \\[1em]
    F_y = m_b (g + \ddot{y}_w)
    = m_b (g - \ddot{\theta}_b l_b \sin\theta_b - \dot{\theta}^2_b l_b \cos\theta_b)
    \\[1em]
    F_t = \frac{T_m - T_f - I_w \ddot{\theta}_w}{l_w} = \frac{T_m - T_f}{l_w} - \frac{I_w}{l^2_w} \ddot{x}_w.
  \end{cases}
\end{equation}
We do not want our system equation to depend on $F_x, F_y, F_t$.
Additionally,
\begin{equation}\label{eq:T_f}
T_f = b_f\left(
\dot{\theta}_w - \dot{\theta}_b
\right) =
b_f\left(
\frac{\dot{x}_w}{l_w} - \dot{\theta}_b
\right).
\end{equation}
We also find the \ac{EOM} for the motor by applying Newton's laws of rotational motion:
\begin{align}
  J_m \ddot{\theta}_m &= K_t i_m = T_m, \quad \text{assuming negligible viscous coefficient;}\nonumber \\
  \Rightarrow i_m &= \frac{J_m \ddot{\theta}_m}{K_t}\label{eq:motor/newton}
\end{align}
and analyse an equivalent circuit which models the motor:
\begin{equation}\label{eq:motor/circuit}
R_m i_m = v_m - K_e \dot{\theta}_m, \quad \text{assuming negligible inductance.}
\end{equation}
Eq.~\eqref{eq:motor/newton} in Eq.~\eqref{eq:motor/circuit} and rearrangement gives us
\begin{equation}\label{eq:motor/circuit2}
  J_m \ddot{\theta}_m
  + \frac{K_t K_e}{R_m} \dot{\theta}_m
  = \frac{K_t}{R_m}v_m.
\end{equation}
But $J_m \ddot{\theta}_m = T_m$, and $\dot{\theta}_m = \frac{\dot{x}_w}{l_w} - \dot{\theta}_b$ in Eq.~\eqref{eq:motor/circuit2} gives us
\begin{equation}\label{eq:motor}
  T_m = \frac{K_t}{R_m}v_m -
  \frac{K_t K_e}{R_m}\left(
    \frac{\dot{x}_w}{l_w}
    - \dot{\theta}_b
  \right).
\end{equation}
We substitute Eqs.~\eqref{eq:F_xyt}--\eqref{eq:T_f}, \eqref{eq:motor} into Eq.~\eqref{eq:system-init} and simplify:
\begin{equation}\label{eq:system}
  \begin{cases}
    \begin{aligned}
      \ddot{x}_w\left(l_w(m_w + m_b) + \frac{I_w}{l_w}\right) =
      \left(
        \frac{K_t K_e}{R_m}
        + b_f
      \right)\dot{\theta}_b
      +
      \left(
        \frac{b_f}{l_w}
        -
        \frac{K_t K_e}{R_m l_w}
      \right)\dot{x}_w
      \\
      + \frac{K_t}{R_m}v_m
      - m_b l_b l_w \ddot{\theta}_b \cos\theta_b
      - m_b l_b l_w \dot{\theta}_b^2 \sin\theta_b
    \end{aligned}\\[2.5em]
    \begin{aligned}
      \ddot{\theta}_b\left(I_b + m_b l_b^2\right)
      =
      -\left(
        \frac{b_f}{l_w}
        +
        \frac{K_t K_e}{R_m}
      \right)\dot{\theta}_b
      + \left(
        \frac{b_f}{l_w} + \frac{K_t K_e}{R_m l_w}
      \right)\dot{x}_w
      \\
      - \frac{K_t}{R_m}v_m
      + m_b l_b g \sin\theta_b
      - m_b l_b \ddot{x}_w.
    \end{aligned}
  \end{cases}
\end{equation}

\subsection*{Reporting of Task 3.2.1}
\begin{enumerate}
\item % Say which linearization point you chose;
  We choose the linearization point of $\theta_b = 0$.
\item % Say why you chose that specific point;
  We chose $\theta_b = 0$ because it is an equalibrium point and because we assume our system will only be subject to small angles.
\item % Write the linearized EOM
  With small-angle approximation, we redefine our non-linear terms in Eq.~\eqref{eq:system}:
  \begin{equation}\label{eq:nonlinear-redef}
    \begin{cases}
      \begin{aligned}
        \sin(\theta_b) &\triangleq \theta_b, \\
        \ddot{x}_w \cos(\theta_b) &\triangleq \ddot{x}_w, \\
        \ddot{\theta}_b \cos(\theta_b) &\triangleq 0, \\
        \dot{\theta}_b^2 \sin(\theta_b) &\triangleq 0.
      \end{aligned}
    \end{cases}
  \end{equation}
  Our linearized \ac{EOM}: Eq.~\eqref{eq:nonlinear-redef}
  in Eq.~\eqref{eq:system} simplifed thus becomes
  \begin{equation}
    \begin{cases}
      \begin{aligned}
        \ddot{x}_w\left(l_w(m_w + m_b) + \frac{I_w}{l_w}\right) =
        \left(
          \frac{K_t K_e}{R_m}
          + b_f
        \right)\dot{\theta}_b
        +
        \left(
          \frac{b_f}{l_w}
          -
          \frac{K_t K_e}{R_m l_w}
        \right)\dot{x}_w
        \\
        + \frac{K_t}{R_m}v_m
        - m_b l_b l_w \ddot{\theta}_b
      \end{aligned}\\[2.5em]
      \begin{aligned}
        \ddot{\theta}_b\left(I_b + m_b l_b^2\right)
        =
        -\left(
          \frac{b_f}{l_w}
          +
          \frac{K_t K_e}{R_m}
        \right)\dot{\theta}_b
        + \left(
          \frac{b_f}{l_w} + \frac{K_t K_e}{R_m l_w}
        \right)\dot{x}_w
        \\
        - \frac{K_t}{R_m}v_m
        + m_b l_b g \theta_b
        - m_b l_b \ddot{x}_w.
      \end{aligned}
    \end{cases}
  \end{equation}
\end{enumerate}

\subsection*{Reporting of Task 3.3.1}
\begin{enumerate}
\item % Say what is your choice for x;
  We define
  \begin{equation*}
    u \triangleq v_m, \quad
    y \triangleq \theta_b
  \end{equation*}
  \begin{equation*}
    \bm{x} \triangleq \begin{bmatrix}
      x_1\\
      x_2\\
      x_3\\
      x_4
    \end{bmatrix}
    \triangleq
    \begin{bmatrix}
      x_w\\
      \dot{x}_w\\
      \theta_b\\
      \dot{\theta}_b
    \end{bmatrix}
    % \Rightarrow
    % \bm{\dot{x}} =
    % \begin{bmatrix}
    %   x_2\\
    %   \ddot{x}_w\\
    %   x_4\\
    %   \ddot{\theta}_b
    % \end{bmatrix}
  \end{equation*}
\item % write the matrices A, B, C, and D in (12) in parametric form;
  Our system in parametric system-state form is the following:
  \begin{equation*}
    \begin{aligned}
      \begin{bmatrix}
        l_w(m_w + m_b) + \frac{I_w}{l_w} &
        l_w l_b m_b \\
        m_b l_b &
        I_b + m_b l_b^2
      \end{bmatrix}
      \begin{bmatrix}
        \ddot{x}_w \\
        \ddot{\theta}_b
      \end{bmatrix}
      =\\
      \begin{bmatrix}
        0 &
        \frac{b_f}{l_w} - \frac{K_t K_e}{R_m l_w} &
        0 &
        \frac{K_t K_e}{R_m} + b_f
        \\[1.2em]
        0 &
        \frac{b_f}{l_w} + \frac{K_t K_e}{R_m l_w} &
        m_b l_b g &
        - \frac{b_f}{l_w} - \frac{K_t K_e}{R_m}
      \end{bmatrix}
      \begin{bmatrix}
        x_w\\
        \dot{x}_w\\
        \theta_b\\
        \dot{\theta}_b
      \end{bmatrix}
      +
      \begin{bmatrix}
        \frac{K_t}{R_m} \\[1.2em]
        -\frac{K_t}{R_m}
      \end{bmatrix}
      u
    \end{aligned}
  \end{equation*}
\item % Write the matrices A, B, C, and D in (12) parametric numeric form (for this purpose use Table 4 on page 64).
  Our system in numeric system-state form is the following:
  \begin{equation}\label{eq:numeric-ss}
    \begin{aligned}
      \begin{bmatrix}
        0.0091 & 0.0009 \\
        0.0427 & 0.0109
      \end{bmatrix}
      \begin{bmatrix}
        \ddot{x}_w \\
        \ddot{\theta}_b
      \end{bmatrix}
      =
      \begin{bmatrix}
        0 & -2.2584 & 0 & 0.0474 \\
        0 & 2.2584 & 0.4182 & -0.0474
      \end{bmatrix}
      \begin{bmatrix}
        x_w\\
        \dot{x}_w\\
        \theta_b\\
        \dot{\theta}_b
      \end{bmatrix}\\
      +
      \begin{bmatrix}
        0.1068 \\
        -0.1068
      \end{bmatrix}
      u
    \end{aligned}
  \end{equation}
\end{enumerate}

\subsection*{Reporting of Task 3.4.1}
\begin{enumerate}
\item % Write the transfer function in the described form
We find ous transfer function to be
\begin{equation}\label{eq:tf}
  G(s) = -90.03\frac{s}{(s + 475)(s + 5.65)(s - 5.72)}
\end{equation}

\item % Describe any numeric MATLAB problems encountered, if any
  The output of \texttt{ss2zp(A, B, C, D)} yields a zero-vector with
  the values of $$z = 10^{-13}\begin{bmatrix} 0 & 0.5684 \end{bmatrix}^T.$$
  We here decide that the non-zero value is sufficiently close to zero
  to be presumed as such. With two zeroes and one pole at $s = 0$ we
  have a pole/zero cancelation that simplifies our transfer function
  to the one in Eq.~\eqref{eq:tf}.
\end{enumerate}

\subsection*{Reporting of Task 3.5.1}
\begin{enumerate}
\item % Describe your choice for the poles, and what you want to achieve in terms of impulse response for the closed loop system;
  Two of our system poles were already negative, namely $s = -475,
  -5.65$, from Eq.~\eqref{eq:tf}. We chose to move our positive, and
  thus unstable pole from $s = 5.72$ to $s = -70$. We chose that pole
  because we want our system to quickly respond to a deviation from
  $\theta_b = 0$. We confirmed the suitability of the pole by
  inspecting the system response via \texttt{impulse(feedback(plant, controller))}.
  Additionally, our poles as real-valued because we do not want an oscillatory system response.
\item % Write the values of the parameters of the PID
  Our found \ac{PID} values are found to be
  \begin{equation}\label{eq:pid}
  K_p = -404.3247, K_i = -2.260 \times 10^3, K_d = -0.8411
  \end{equation}
\item % Write the resulting closed-loop function in its numerical form
  The transfer function of the closed-loop system is the following:
  $$
  \frac{\Theta_b(s)}{V(s)} =
  \frac{-90s}{
    s^3
    + 550.7s^2
    + 3.634 \times 10^4 s
    + 1.881 \times 10^5
  }.
  $$
\end{enumerate}

\subsection*{Reporting of Task 3.6.1}
\begin{enumerate}
\item % Write the novel EOM in parametric form
  We redefine
  $$
  F_x \triangleq m_b\left(
    \ddot{x}_w
    + \ddot{\theta}_b l_b \cos\theta_b
    - \dot{\theta}^2_b l_b \sin\theta_b
  \right)
  - d,
  $$
  and derive our novel \ac{EOM}:
  \begin{equation}\label{eq:system-novel}
    \begin{cases}
      \begin{aligned}
        \ddot{x}_w\left(l_w(m_w + m_b) + \frac{I_w}{l_w}\right) =
        \left(
          \frac{K_t K_e}{R_m}
          + b_f
        \right)\dot{\theta}_b
        +
        \left(
          \frac{b_f}{l_w}
          -
          \frac{K_t K_e}{R_m l_w}
        \right)\dot{x}_w
        \\
        + \frac{K_t}{R_m}v_m
        - m_b l_b l_w \ddot{\theta}_b \cos\theta_b
        - m_b l_b l_w \dot{\theta}_b^2 \sin\theta_b
        + l_w d
      \end{aligned}\\[2.5em]
      \begin{aligned}
        \ddot{\theta}_b\left(I_b + m_b l_b^2\right)
        =
        -\left(
          \frac{b_f}{l_w}
          +
          \frac{K_t K_e}{R_m}
        \right)\dot{\theta}_b
        + \left(
          \frac{b_f}{l_w} + \frac{K_t K_e}{R_m l_w}
        \right)\dot{x}_w
        \\
        - \frac{K_t}{R_m}v_m
        + m_b l_b g \sin\theta_b
        - m_b l_b \ddot{x}_w
        + l_b d.
      \end{aligned}
    \end{cases}
  \end{equation}
\item % Write the novel linearized EOM in state space in numerical form
  Our linearized state space of the \ac{EOM} in numerical form is Eq.~\eqref{eq:numeric-ss} with an additional system input $d$:
  \begin{equation*}
    \begin{aligned}
      \begin{bmatrix}
        0.0091 & 0.0009 \\
        0.0427 & 0.0109
      \end{bmatrix}
      \begin{bmatrix}
        \ddot{x}_w \\
        \ddot{\theta}_b
      \end{bmatrix}
      =
      \begin{bmatrix}
        0 & -2.2584 & 0 & 0.0474 \\
        0 & 2.2584 & 0.4182 & -0.0474
      \end{bmatrix}
      \begin{bmatrix}
        x_w\\
        \dot{x}_w\\
        \theta_b\\
        \dot{\theta}_b
      \end{bmatrix}
      \\
      +
      \begin{bmatrix}
        0.1068 \\
        -0.1068
      \end{bmatrix}
      u
      +
      \begin{bmatrix}
        0.008 \\
        0.043
      \end{bmatrix}
      d
    \end{aligned}
  \end{equation*}
\end{enumerate}
\subsection*{Reporting of Task 3.7.1}
\begin{enumerate}
\item % Plot your simulink scheme
  Our Simulink scheme for our linearized system in
  Eq.~\eqref{eq:system-novel} controlled by Eq.~\eqref{eq:pid} can be seen in Fig.~\ref{fig:simulink-scheme}.
  \begin{figure}
    \includegraphics[width=\linewidth]{Images/3.7.1linearizedBot-system.pdf}
    \caption{Simulink scheme for the linearized system, where the
      f-suffix denote the novel matrices.}
    \label{fig:simulink-scheme}
  \end{figure}
\item % plot realizations of ...
  Our realizations of $\theta_b(t)$, $x_w(t)$, $v_m(t)$ and $d(t)$ can
  be seen in Figs.~\ref{fig:lin-theta_b},
  \ref{fig:lin-x_w},
  \ref{fig:lin-v_m},
  \ref{fig:lin-d},
  respectively.
  \begin{figure}
    \centering
    \begin{tabular}{cc}
        \subcaptionbox{Realization of $\theta_b(t)$.\label{fig:lin-theta_b}}{\includegraphics[width=0.5\linewidth]{Images/3.7.1-theta_b.pdf}}&
        \subcaptionbox{Realization of $x_w(t)$.\label{fig:lin-x_w}}{\includegraphics[width=0.5\linewidth]{Images/3.7.1-x_w.pdf}}
      &
        \subcaptionbox{Realization of $v_m(t)$.\label{fig:lin-v_m}}{\includegraphics[width=0.5\linewidth]{Images/3.7.1-v_m.pdf}}
      &
        \subcaptionbox{Realization of $d(t)$.\label{fig:lin-d}}{\includegraphics[width=0.5\linewidth]{Images/3.7.1-d.pdf}}
      \\
    \end{tabular}
    \caption{Realizations of the system in Fig.~\ref{fig:simulink-scheme}.}
  \end{figure}
\end{enumerate}

\subsection*{Reporting of Task 3.8.1}
\begin{enumerate}
\item We find the bandwidth of our system via \texttt{gh = plant *
    controller; bandwidth(gh / (1 + gh))} to be
  $\omega_b = 69.8340~\text{rad/s} = 11.1144~\text{Hz}$.
\item We choose our samping time to be
  $25\omega_b \Rightarrow T = 0.0036~\text{s}$.
\item Following the recommendations of §8.3.6, we use a scalar value
  of $25$ so that we may ``[use] discrete equivalents [...] with
  confidence''.
\item
  Our discrete controller, $C(z)$, derived via \texttt{c2d(pid(Kp, Ki,
    Kd, T), T, ’zoh’)}\footnote{We have no idea what value of
    \texttt{Tf} we should pass to \texttt{c2d}, but we assume it has
    some relation to $T$.} is
  \begin{equation}
    C(z) =
    K_p
    + K_i\frac{T_s}{z - 1}
    + K_d\left(
      T_f + \frac{T_s}{z - 1}
    \right)^{-1}
    \text{, where}
    \quad
    \begin{cases}
      K_p = -404 \\
      K_i = -2.26 \times 10^{3} \\
      K_d = -1.33 \\
      T_f = 0.00569 \\
      T_s = T
    \end{cases}
  \end{equation}
\end{enumerate}

\subsection*{Reporting of Task 3.9.1}
\begin{enumerate}
\item % report \bar{d}
  We found that $\bar{d} = 1.54$ for both the discrete and
  continuous case. We presume this to be the expected results on the
  basis that we use $25\omega_b$ as our sampling rate.
\item % plot realizations of cont. and disc.
  Our realizations for $\theta_b(t)$, $v_m(t)$, $\theta^{lin}_b(t)$
  and $v^{lin}_m(t)$ where a continuous controller is used can be seen
  in Figs.~
  \ref{fig:cont-theta_b},
  \ref{fig:cont-v_m},
  \ref{fig:cont-lin-theta_b},
  \ref{fig:cont-lin-v_m},
  respectively.
  \begin{figure}
    \centering
    \begin{tabular}{cc}
        \subcaptionbox{Realization of $\theta_b(t)$.\label{fig:cont-theta_b}}{\includegraphics[width=0.5\linewidth]{Images/3.9.1-theta_b-cont-nonlin.pdf}}&
        \subcaptionbox{Realization of $v_m(t)$.\label{fig:cont-v_m}}{\includegraphics[width=0.5\linewidth]{Images/3.9.1-v_m-cont-nonlin.pdf}}
      &
        \subcaptionbox{Realization of $\theta^{lin}_b(t)$.\label{fig:cont-lin-theta_b}}{\includegraphics[width=0.5\linewidth]{Images/3.9.1-theta_b-cont-lin.pdf}}
      &
        \subcaptionbox{Realization of $v^{lin}_m(t)$.\label{fig:cont-lin-v_m}}{\includegraphics[width=0.5\linewidth]{Images/3.9.1-v_m-cont-lin.pdf}}
      \\
    \end{tabular}
    \caption{Realizations using a continuous controller.}
  \end{figure}

  Our realizations for $\theta_b(k)$, $v_m(k)$, $\theta^{lin}_b(k)$
  and $v^{lin}_m(k)$ where a discrete controller is used can be seen
  in Figs.
  \ref{fig:disc-theta_b},
  \ref{fig:disc-v_m},
  \ref{fig:disc-lin-theta_b},
  \ref{fig:disc-lin-v_m},
  respectively.
    \begin{figure}
    \centering
    \begin{tabular}{cc}
        \subcaptionbox{Realization of $\theta_b(k)$.\label{fig:disc-theta_b}}{\includegraphics[width=0.5\linewidth]{Images/3.9.1-theta_b-disc-nonlin.pdf}}&
        \subcaptionbox{Realization of $v_m(k)$.\label{fig:disc-v_m}}{\includegraphics[width=0.5\linewidth]{Images/3.9.1-v_m-disc-nonlin.pdf}}
      &
        \subcaptionbox{Realization of $\theta^{lin}_b(k)$.\label{fig:disc-lin-theta_b}}{\includegraphics[width=0.5\linewidth]{Images/3.9.1-theta_b-disc-lin.pdf}}
      &
        \subcaptionbox{Realization of $v^{lin}_m(k)$.\label{fig:disc-lin-v_m}}{\includegraphics[width=0.5\linewidth]{Images/3.9.1-v_m-disc-lin.pdf}}
      \\
    \end{tabular}
    \caption{Realizations using a discrete controller.}
  \end{figure}
\item % discuss what was understood from these expiremints
  We did not iterate on the \ac{PID} parameters during or after simulating
  because our realizations showed satisfactory results.
\end{enumerate}

\end{document}
