\documentclass[11pt]{article} % ~~~~~~~~~~~~~~~~~~~~~~~ %
%														%
\input{./Scripts/packages}								%
\input{./Scripts/ridefinitions}							%
\input{./Scripts/figuresgraphicalsettings}				%
\input{./Scripts/tablesgraphicalsettings}				%
\input{./Scripts/newcommands}							%
\input{./Scripts/colors}								%
%														%
% ~~~~~~~~~~~~~~~~~~~~~~~~~~~~~~~~~~~~~~~~~~~~~~~~~~~~~ %

\usepackage{mathtools}
\usepackage{subcaption}

\title{\Huge R7003E 2017 LP2 \\ Lab B report \\ Group 3}
\date{\today}
\author{                        % In alphabetical order, family name first
  Andersson Tommy \\
  Graden Samuel \\
  Sonesten Viktor
}


\begin{documen}
\maketitle

% \begin{instructions}
% 	%
% 	For this lab report there is no size limit on your report, but try to be concise. As for the check-list, put a \checkmark when you have done the indicated things. Before submitting the report you mush have \checkmark-marked every row. Delete also these instructions and all the footnotes before submitting.
% 	%
% \end{instructions}

\section*{Check-list}

\begin{center}
	\rowcolors{1}{black!05!white}{}
	\begin{tabular}{|m{0.8\columnwidth}>{\centering \arraybackslash}m{0.1\columnwidth}|}
		\hline
		\checkmark the fonts in the figures are the same fonts in the
          rest of the text & \\%\footnote{Suggestion: if you use Matlab then take a look at \texttt{http://staff.www.ltu.se/{\textasciitilde}damvar/matlab.html}. If you instead want to do things seriously in \LaTeX\ then consider using the package \texttt{pgfplots}.} & \\
		\checkmark the captions of the figures are self-readable and terminate with a dot & \\
		\checkmark the figures are color-blind-people-friendly and do not have useless backgrounds & \\
		\checkmark colors are used to convey information & \\% \footnote{And not to make figures ``more colored''. E.g., red should be used to indicate something ``wrong'' or ``important'', green something ``correct'', etc.} & \\
		\checkmark all the figures that are not photos are in vectorial format & \\
		\checkmark the figures have meaningful legends that allow to understand what is what & \\
		\checkmark all the acronyms are defined properly &\\ % \footnote{Suggestion: use the package \texttt{acronym}.} & \\
		\checkmark there are no spelling errors & \\
		\checkmark what needs to be referenced is referenced & \\
		\checkmark the parentheses in the various equations are as big as needed & \\
		\checkmark the fonts used in the equations are sufficiently big to be readable comfortably & \\
		\hline
	\end{tabular}
\end{center}


\begin{acronym}[TDMA]
	\acro{EOM}	[EOM]	{Equations of Motion}
	% \acro{DC}	[DC]	{Direct Current}
	% \acro{SS}	[SS]	{State-Space}
	\acro{PID}	[PID]	{Proportional Integrative Derivative}
	% \acro{LQR}	[LQR]	{Linear-Quadratic Regulator}
	% \acro{emf}	[emf]	{electromotive force}
\end{acronym}
% basic usage: \ac{EOM} \acf{EOM} \acl{EOM}

\subsection*{Reporting of Task 4.4.1}

\end{document}
